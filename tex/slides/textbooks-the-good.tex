\begin{frame}[fragile]{Julia for Textbooks: \textcolor{paloalto}{The Good}} \pause

\begin{itemize}
  \item \textcolor{paloalto}{Concise} algorithm descriptions (fits within single page) \pause
  \item \textcolor{paloalto}{Multiple dispatch} makes it easy to design common interface \pause
  \item \textcolor{paloalto}{Auto-differentiation} allows for clean algorithms that just work \pause
  \item \textcolor{paloalto}{Integrated tooling} to run and plot Julia code during PDF compilation \pause
  \item \textcolor{paloalto}{Full Unicode support} means math matches code (e.g., $\lambda$ in math and \jlv{λ} in code) \pause
  \item \textcolor{paloalto}{Seamless package integration} to not reinvent the wheel
\end{itemize}

\pause

\phantom{---}

\small
\shortstack[l]{\textbf{Shout-out to the following packages}:\\\jlv{Distributions.jl}, \jlv{LazySets.jl}, \jlv{IntervalArithmetic.jl}, \jlv{Flux.jl}, \jlv{Optim.jl}, and \jlv{JuMP.jl}}

\end{frame}