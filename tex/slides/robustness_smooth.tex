% Note, [fragile] is needed when using juliaconsole
\begin{frame}[fragile,t]
\frametitle{Robustness and Smooth Robustness}
\framesubtitle{Example: Eventually $\lozenge$}

\phantom{}

\begin{small}
\textit{First, let's define the \jlv{Eventually} operator in Julia.}
\end{small}

\phantom{}

\pause
{\fontsize{8}{10}\selectfont
\begin{algorithmblock}
\begin{juliaverbatim}
mutable struct Eventually <: Formula
    ψ::Formula
    I::Interval
end
\end{juliaverbatim}
\end{algorithmblock}}

\begin{footnotesize}
\begin{align*}
    \onslide<3->{\lozenge_{[a,b]}\psi &= \top \mathcal{U}_{[a,b]} \psi \\}
                         \onslide<4->{&= \exists t (a \le t \le b) \psi_t}
\end{align*}
\end{footnotesize}

\pause\pause\pause
{\fontsize{8}{10}\selectfont
\begin{algorithmblock}
\begin{juliaverbatim}
(◊::Eventually)(x) = any(◊.ψ(x[t]) for t ∈ interval(◊,x)) # Evaluate formula ψ(x)
\end{juliaverbatim}
\end{algorithmblock}}

\end{frame}

%%%%%%%%%%%%%%%%%%%%%%%%%%%%%%%%%%%%%%%%

\begin{frame}[fragile,t]
\frametitle{Robustness and Smooth Robustness}
\framesubtitle{Example: Eventually $\lozenge$}

\pause
\begin{scriptsize}
\begin{equation*}
    \rho(x_t, \lozenge_{[a,b]}\psi) = \max_{t^\prime \in [t+a,t+b]} \rho(x_{t^\prime}, \psi)
\end{equation*}
\end{scriptsize}
\vspace{-0.5\baselineskip}
\pause
{\fontsize{8}{10}\selectfont
\begin{algorithmblock}
\begin{juliaverbatim}
ρ(x, ◊::Eventually) = maximum(ρ(x[t], ◊.ψ) for t ∈ interval(◊,x)) # Robustness
\end{juliaverbatim}
\end{algorithmblock}}

\vspace{\baselineskip}

\pause
\begin{scriptsize}
\begin{equation*}
    \tilde{\rho}(x_t, \lozenge_{[a,b]}\psi) = \operatorname*{\widetilde{\max}}_{t^\prime \in [t+a,t+b]} \tilde{\rho}(x_{t^\prime}, \psi)
\end{equation*}
\end{scriptsize}
\vspace{-0.5\baselineskip}
\pause
{\fontsize{8}{10}\selectfont
\begin{algorithmblock}
\begin{juliaverbatim}
ρ̃(x, ◊::Eventually; w=1) = smoothmax(ρ̃(x[t], ◊.ψ; w) for t ∈ interval(◊,x); w) # Smooth robustness
\end{juliaverbatim}
\end{algorithmblock}}

\vspace{\baselineskip}

\pause
\begin{scriptsize}
\begin{equation*}
    \widetilde{\max}(x; w) = \frac{\sum_i^n x_i \exp(x_i/w)}{\sum_j^n \exp(x_j/w)}
\end{equation*}
\end{scriptsize}
\vspace{-0.5\baselineskip}
\pause
{\fontsize{8}{10}\selectfont
\begin{algorithmblock}
\begin{juliaverbatim}
smoothmax(x; w=1) = (w == 0) ? maximum(x) : sum(xᵢ*exp(xᵢ/w) for xᵢ in x) / sum(exp(xⱼ/w) for xⱼ in x)
\end{juliaverbatim}
\end{algorithmblock}}

\vspace{\baselineskip}

\end{frame}

%%%%%%%%%%%%%%%%%%%%%%%%%%%%%%%%%%%%%%%%

% \begin{figure}
    \begin{jlcode}
    p = let
        times = collect(1:10)
        τ = [-1.0, -3.2, 2.0, 1.5, 3.0, 0.5, -0.5, -2.0, -4.0, -1.5];
        ψ = @formula ◊(sₜ -> sₜ > 0);
        ∇ = gradient(τ->ρ(τ, ψ), τ)
        ∇̃ = gradient(τ->ρ̃(τ, ψ), τ)

        xmin, xmax, ymin, ymax = 1, 10, -5, 4
        # xmin, xmax, ymin, ymax = 1, 10, -0.1, 1.1
        ax = Axis(xmin=xmin, xmax=xmax, ymin=ymin, ymax=ymax, width="7cm", height="4cm")
        ax.xlabel = "Time"
        ax.ylabel = L"s"
        # ax.title = raw"${\color{pastelSkyBlue}\psi_1 = \lozenge\big( s_t > 0 \big)} \qquad {\color{pastelGreen}\psi_2 = \square\big( s_t > 0 \big)}$"
        ax.style = raw"title style={font=\footnotesize}"

        # Used to add phantom right yaxis for better centering
        axr = deepcopy(ax)
        axr.style *= raw", axis y line*=right, axis x line=none, yticklabel={\phantom{\pgfmathprintnumber{\tick}}}"
        axr.ylabel = "\\phantom{$(ax.ylabel)}"
        push!(axr, Plots.Command(""))

        push!(ax, Plots.Linear(times, τ, style="solid, mark=*, gray, line width=1pt, mark options={fill=gray!50, draw=gray}, mark size=2pt"))
        push!(ax, Plots.Linear([xmin, xmax], [0, 0], style="dotted, mark=none, gray, line width=1pt"))

        push!(ax, Plots.Linear(times, ∇, style="solid, mark=*, pastelGreen, line width=1pt, mark options={fill=pastelGreen!50, draw=pastelGreen}, mark size=2pt"))
        push!(ax, Plots.Linear(times, ∇̃, style="solid, mark=*, pastelBlue, line width=1pt, mark options={fill=pastelBlue!50, draw=pastelBlue}, mark size=2pt"))

        # push!(ax, Plots.Scatter([5], [3], style="only marks, mark=*, mark size=2pt, mark options={fill=pastelSkyBlue!50, draw=pastelSkyBlue}"))
        # push!(ax, Plots.Linear([xmin, xmax], [3, 3], style="dashed, mark=none, pastelSkyBlue, line width=1pt"))
        # push!(ax, Plots.Command("\\node[anchor=center] at (axis cs: 8, 2.3) {\\scriptsize \\textcolor{pastelSkyBlue}{\$\\rho_1\$}};"))

        # push!(ax, Plots.Scatter([9], [-4], style="only marks, mark=*, mark size=2pt, mark options={fill=pastelGreen!50, draw=pastelGreen}"))
        # push!(ax, Plots.Linear([xmin, xmax], [-4, -4], style="dashed, mark=none, pastelGreen, line width=1pt"))
        # push!(ax, Plots.Command("\\node[anchor=center] at (axis cs: 5, -3.3) {\\scriptsize \\textcolor{pastelGreen}{\$\\rho_2\$}};"))

        [axr, ax]
    end
    plot(p)
    \end{jlcode}
    \plotlater{fig/jl_robustness_gradient}
\end{figure}

\begin{frame}[fragile,t]
\frametitle{Robustness and Smooth Robustness}
\framesubtitle{Example: Eventually $\lozenge$}

\phantom{}

\begin{small}
\textit{This means we can use Julia's {\color{captioncolor}auto-differentiation} packages right out-of-the-box!}
\end{small}

\pause
\begin{columns}
\begin{column}[t]{0.5\textwidth}
\begin{tiny}
\begin{juliaconsole}
using SignalTemporalLogic
import ForwardDiff: gradient
τ = [-1.0, -3.2, 2.0, 1.5, 3.0, 0.5, -0.5, -2.0, -4.0, -1.5];
ψ = @formula ◊(sₜ -> sₜ > 0);
gradient(τ->ρ(τ, ψ), τ)
\end{juliaconsole}
\onslide<2>{\lineblackout{11}}
\onslide<2-3>{\lineblackout{0}\lineblackout{1}\lineblackout{2}\lineblackout{3}\lineblackout{4}\lineblackout{5}\lineblackout{6}\lineblackout{7}\lineblackout{8}\lineblackout{9}\lineblackout{10}}
\end{tiny}
\end{column}

\pause\pause\pause
\begin{column}[t]{0.5\textwidth}
\begin{tiny}
\begin{juliaconsole}
using SignalTemporalLogic; import ForwardDiff: gradient
τ = [-1.0, -3.2, 2.0, 1.5, 3.0, 0.5, -0.5, -2.0, -4.0, -1.5];
ψ = @formula ◊(sₜ -> sₜ > 0);
# Smooth robustness gradient
gradient(τ->ρ̃(τ, ψ), τ)
\end{juliaconsole}
% \lineblackout{12}
\lineblackout{13}\lineblackout{14}\lineblackout{15}
\onslide<1-6>{\lineblackout{0}\lineblackout{1}\lineblackout{2}\lineblackout{3}\lineblackout{4}\lineblackout{5}\lineblackout{6}\lineblackout{7}\lineblackout{8}\lineblackout{9}\lineblackout{10}}
\onslide<7>{}

\phantom{}

\pause
\jlv{ρ̃ # \rho<TAB>\tilde<TAB>}
\end{tiny}
\end{column}
\end{columns}

% \pause
% \begin{tikzpicture}[remember picture, overlay]
%     \node[anchor=south west] at ($(current page.south west)+(1.125cm,0)$) {
%         \input{fig/jl_robustness_gradient} % Use \input not \plot when reusing plots
%     };
% \end{tikzpicture}

\end{frame}


%%%%%%%%%%%%%%%%%%%%%%%%%%%%%%%%%%%%%%%%


\begin{frame}[fragile,t]
\frametitle{Robustness and Smooth Robustness}
\framesubtitle{Example: Eventually $\lozenge$}

\phantom{}

\begin{small}
\textit{The textbook (and projects) use aliases for \jlv{ρ} and \jlv{ρ̃}}
\end{small}

\phantom{}

\pause
\begin{juliaverbatim}
robustness(𝐬, ψ) = ρ(𝐬, ψ)
\end{juliaverbatim}

\begin{juliaverbatim}
smooth_robustness(𝐬, ψ; w=1) = ρ̃(𝐬, ψ; w=1)
\end{juliaverbatim}

\end{frame}