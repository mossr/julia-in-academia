\begin{frame}[fragile]{Assignments in Julia} \pause

\begin{columns}
  \begin{column}{0.45\textwidth}
    \centering
    \includegraphics[width=\linewidth]{media/AA228VProjects.png}

    \captionof*{figure}{\shortstack{\footnotesize Assignments repository\\\textcolor{gray}{\scriptsize Student-facing code}}}
  \end{column}
  \pause
  \begin{column}{0.45\textwidth}
    \centering
    \includegraphics[width=\linewidth]{media/project2.png}

    \captionof*{figure}{\shortstack{\footnotesize Pluto assignments\\\textcolor{gray}{\scriptsize Includes local tests}}}
  \end{column}
  \hfill
\end{columns}

\end{frame}


\begin{frame}[fragile]{Assignments in Julia}

\begin{columns}
  \hfill
  \begin{column}{0.45\textwidth}
    \centering
    \includegraphics[width=\linewidth]{media/project1.png}

    \captionof*{figure}{\shortstack{\footnotesize Interactive Pluto tests\\\textcolor{gray}{\scriptsize Get feedback instantly}}}
  \end{column}
  \pause
  \begin{column}{0.54\textwidth}
    \begin{itemize}
      \item Separated assignment code from core library (\jlv{StanfordAA228V.jl}) \pause
      \item Local tests exactly match graded tests \pause
      \item Interactive nature allows students to play with their algorithms \pause
      \item No ``hidden state'' that may confuse students \pause
      \item Open-source nature requires clever obfuscation of solution code
    \end{itemize}
  \end{column}
  \hfill
\end{columns}

\end{frame}