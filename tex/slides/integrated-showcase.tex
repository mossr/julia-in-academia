\begin{frame}[fragile]{Showcase: \normalfont\jlv{IntervalArithmetic.jl} and \jlv{LazySets.jl}} \pause

{\scriptsize
\begin{algorithmblock}
\begin{juliaverbatim}
struct NaturalInclusion <: ReachabilityAlgorithm
    h # time horizon
end

function r(sys, x)
    s, 𝐱 = extract(sys.env, x)
    τ = rollout(sys, s, 𝐱)
    return τ[end].s
end

to_hyperrectangle(𝐈) = Hyperrectangle(low=[i.lo for i in 𝐈], 
                                      high=[i.hi for i in 𝐈])

function reachable(alg::NaturalInclusion, sys)
    𝐈′s = []
    for d in 1:alg.h
        𝐈 = intervals(sys, d)
        push!(𝐈′s, r(sys, 𝐈))
    end
    return UnionSetArray([to_hyperrectangle(𝐈′) for 𝐈′ in 𝐈′s])
end
\end{juliaverbatim}
\end{algorithmblock}
}

\end{frame}


\begin{frame}[fragile]{Showcase: \normalfont\jlv{IntervalArithmetic.jl} and \jlv{LazySets.jl}}

\begin{figure} % interval counterparts
    \begin{jlcode}
    p = let
        xmin, xmax, ymin, ymax = 0, 2.75, 0, 10
        ax1 = plot_interval(exp, interval(1, 2), interval(exp(1), exp(2)), xmin, xmax, ymin, ymax, fxmax=log(ymax), color="pastelSkyBlue")
        ax1.xlabel = L"x"
        ax1.ylabel = L"f(x)"
        ax1.title = L"f(x) = \exp(x)"
        ax1.height = "4.5cm"
        ax1.width = "4.5cm"

        xmin, xmax, ymin, ymax = -2, 2, 0, 4.5
        ax2 = plot_interval(x->x^2, interval(-1.5, 1), interval(0, 1.5^2), xmin, xmax, ymin, ymax, color="pastelSkyBlue")
        ax2.xlabel = L"x"
        ax2.title = L"\phantom{\sin(x)}f(x) = x^2\phantom{\sin(x)}"
        ax2.height = "4.5cm"
        ax2.width = "4.5cm"

        xmin, xmax, ymin, ymax = -3.14, 3.14, -1.5, 1.5
        ax3 = plot_interval(sin, interval(0, 2), interval(0, 1), xmin, xmax, ymin, ymax, color="pastelSkyBlue")
        ax3.xlabel = L"x"
        ax3.title = L"f(x) = \sin(x)"
        ax3.height = "4.5cm"
        ax3.width = "4.5cm"

        g = GroupPlot(3, 1, groupStyle="horizontal sep=1cm")
        push!(g, ax1)
        push!(g, ax2)
        push!(g, ax3)
        g
    end
    plot(p)
    \end{jlcode}
    \begin{center}
        \plot{fig/interval_counterparts}
    \end{center}
	\caption{Example of the interval counterparts for the $\exp$, square, and $\sin$ functions.}
\end{figure}

\end{frame}


\begin{frame}[fragile]{Showcase: \normalfont\jlv{IntervalArithmetic.jl} and \jlv{LazySets.jl}}

{\small
\begin{algorithmblock}
\begin{juliaverbatim}
function (env::InvertedPendulum)(s, a)
    θ, ω = s[1], s[2]
    dt, g, m, l = env.dt, env.g, env.m, env.l
    ω = ω + (3g / (2 * l) * sin(θ) + 3 * a / (m * l^2)) * dt
    θ = θ + ω * dt
    return [θ, ω]
end
\end{juliaverbatim}
\end{algorithmblock}
}

\end{frame}


\begin{frame}[fragile]{Showcase: \normalfont\jlv{IntervalArithmetic.jl} and \jlv{LazySets.jl}}
    
\begin{figure}
    \begin{jlcode}
    function intervals(sys, d)
        disturbance_mag = 0.01
        θmin, θmax = -π/16, π/16
        ωmin, ωmax =  -1.0, 1.0
        𝐈 = [interval(θmin, θmax), interval(ωmin, ωmax)]
        for i in 1:d
            push!(𝐈, interval(-disturbance_mag, disturbance_mag))
            push!(𝐈, interval(-disturbance_mag, disturbance_mag))
        end
        return 𝐈
    end
    function extract(env::InvertedPendulum, x)
        s = x[1:2]
        𝐱 = [Disturbance(0, 0, x[i:i+1]) for i in 3:2:length(x)]
        return s, 𝐱
    end
    p = let
        agent = ProportionalController([-15., -8.])
        env = InvertedPendulum()
        disturbance_mag = 0.01
        θmin, θmax = -π/16, π/16
        ωmin, ωmax =  -1.0, 1.0
        sensor = AdditiveNoiseSensor(Product([Uniform(-disturbance_mag, disturbance_mag), Uniform(-disturbance_mag, disturbance_mag)]))
        Ps(env::InvertedPendulum) = Product([Uniform(θmin, θmax), Uniform(ωmin, ωmax)])
        inverted_pendulum = System(agent, env, sensor)

        d = 2
        Random.seed!(4)
        τs = [rollout(inverted_pendulum, d=d) for i in 1:150]
        sfinals = [τ[end].s for τ in τs]

        alg = NaturalInclusion(d)
        ℛ = reachable(alg, inverted_pendulum)

        ax = inv_pendulum_state_space(ymin=-3, ymax=3)
        push!(ax, Plots.Scatter([s[1] for s in sfinals], [s[2] for s in sfinals], 
            style="solid, mark=*, mark size=1pt, mark options={draw=pastelSkyBlue, fill=pastelSkyBlue!80, fill opacity=1.0, draw opacity=1.0}"))
        plot_polytope!(ax, ℛ[2], fill_color="pastelPurple!20!black", draw_color="pastelPurple", fill_alpha=1.0, draw_alpha=1.0)
        ax.height = "4.5cm"
        ax.width = "4.5cm"
        ax
    end
    plot(p)
    \end{jlcode}
    \begin{center}
        \plot{fig/natural_inclusion_pendulum}
    \end{center}
    \caption{Inverted pendulum reachable set using \jlv{IntervalArithmetic.jl} and \jlv{LazySets.jl}.}
\end{figure}

\end{frame}
