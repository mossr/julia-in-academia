% [fragile] needed when the content contains juliaverbatim.
\begin{frame}[fragile]{TikZ figures}

For a neural network with one hidden layer, the intermediate hidden units are $h_j = \sigma(\v_j \cdot \phi(x))$ where $\sigma(z) = (1 + e^{-z})^{-1}$, producing output $\text{score} = \w \cdot \h$:

\begin{figure}
\begin{center}
\begin{tikzpicture}
    \node (phi2) [nnnode, label=left:{$\phi(x)_2$}] {};
    \node (phi1) [nnnode, above=2mm of phi2, label=left:{$\phi(x)_1$}] {};
    \node (phi3) [nnnode, below=2mm of phi2, label=left:{$\phi(x)_3$}] {};

    \node (hidden1) [nnnode, above right=0mm and 15mm of phi2, label=above:{$h_1$}] {$\sigma$};
    \node (hidden2) [nnnode, below right=0mm and 15mm of phi2, label=below:{$h_2$}] {$\sigma$};

    \node (score) [nnnode, right=30mm of phi2, label=above:{score}] {};

    \draw[->] (phi1) -- (hidden1) node [midway, yshift=7pt] {$\darkred{\V}$};
    \draw[->] (hidden1) -- (score) node [midway, yshift=7pt] {$\darkred{\w}$};
    \draw[->] (phi1) -- (hidden2);
    \draw[->] (phi2) -- (hidden1);
    \draw[->] (phi2) -- (hidden2);
    \draw[->] (phi3) -- (hidden1);
    \draw[->] (phi3) -- (hidden2);
    \draw[->] (hidden1) -- (score);
    \draw[->] (hidden2) -- (score);
\end{tikzpicture}
\end{center}
\caption{
    \scriptsize
    \label{fig:nn}
    A \textit{one-layer neural network} with $\phi(x) \in \R^3$ and $\h \in \R^2$ using the logistic activation function $\sigma$.
}
\end{figure}

\end{frame}