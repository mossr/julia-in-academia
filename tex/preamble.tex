\newcommand*{\INVERTED}{} % Comment out for light-mode
\newcommand*{\CENTERTITLE}{} % Comment out for left-aligned title

\usepackage{fontspec}
\usepackage{sourcesanspro}
\usepackage[T1]{fontenc}

% NOTE: You will need to pull this code via the GitHub repo and compile with latexmk
% to get new pythontex syntax highlighting working properly.
% \documentclass[aspectratio=169]{beamer-pytex}

\setbeamertemplate{title page}[default][left] % left-align title page (commend out for centered title page)
\beamertemplatenavigationsymbolsempty % remove navigation symbols

\newcommand*{\INVERTED}{} % Comment out for light-mode
\newcommand*{\CENTERTITLE}{} % Comment out for left-aligned title

\usepackage{fontspec}
\usepackage{sourcesanspro}
\usepackage[T1]{fontenc}

% NOTE: You will need to pull this code via the GitHub repo and compile with latexmk
% to get new pythontex syntax highlighting working properly.
% \documentclass[aspectratio=169]{beamer-pytex}

\setbeamertemplate{title page}[default][left] % left-align title page (commend out for centered title page)
\beamertemplatenavigationsymbolsempty % remove navigation symbols

\input{preamble}

\title{\texttt{SignalTemporalLogic.jl}}
\subtitle{Introduction}
\author{Robert Moss}
\institute{Stanford AA228V/CS238V}
\email{mossr@cs.stanford.edu}
\date{}

\begin{document}

\begin{frame}
    \maketitle
\end{frame}

\include{slides/pkg}
\include{slides/spec}
\include{slides/spec2}
\include{slides/robustness}
\include{slides/course}
\include{slides/gridworld}
\include{slides/problem-cw}
\include{slides/problem-pendulum}
\include{slides/problem-cas}
\include{slides/resources}

\end{document}

% TODO: albatian
\usefonttheme{serif}
\usefonttheme{professionalfonts}
\usepackage{mathtools}
\usepackage[tracking=true]{microtype}

\usepackage{bold-extra}
\usepackage{realscripts}
\usepackage{amsmath,bm,amssymb}
\usepackage{amsthm}
\usepackage{bbm}
\usepackage{tikz}
\usepackage[group-digits=integer,group-minimum-digits=4,group-separator={,},detect-all]{siunitx}
\usepackage{algorithmicx}
\usepackage{algorithm}
\usepackage[noend]{algpseudocode}
\usepackage{xcolor}
\usepackage{multirow}
\usepackage{longtable,tabularx,booktabs}
\usepackage[flushleft]{threeparttable}
\usepackage{vector} % local
\usepackage{varwidth}
\usepackage{fancyvrb}
\usepackage{tcolorbox}
\usepackage{ifthen}
\usepackage{pagecolor}
\usepackage{textpos}
\usepackage{calc}
\usepackage{adjustbox}
\usepackage{svg}
\usepackage{caption}
\usepackage{verbatimbox}
\usepackage{multimedia}
\usepackage{xmpmulti}

%%%%%%%%%%%%%%%%%%%%%%%%%%%%%%%%%%%%%%%%%%%%%%%%%%
% Unicode settings
%%%%%%%%%%%%%%%%%%%%%%%%%%%%%%%%%%%%%%%%%%%%%%%%%%
% \setmonofont{DejaVu Sans Mono}[Scale=MatchLowercase]
% \usepackage{newunicodechar}
% \newfontface{\calligraphic}{Latin Modern Math}[Scale=0.85]

% https://juliamono.netlify.app/faq/#can_i_use_it_in_a_latex_document
\setmonofont{JuliaMono-Regular}[
	UprightFont = JuliaMono-Medium,
	BoldFont = JuliaMono-ExtraBold,
	ItalicFont = JuliaMono-Light,
    Contextuals = Alternate,
    Scale = MatchLowercase,
    Path = ./fonts/,
    Extension = .ttf
]

% Remove empty lines in juliaconsole
% https://github.com/gpoore/pythontex/issues/98#issuecomment-318078264
\makeatletter
\AtBeginEnvironment{juliaconsole}{%
  \def\FV@@PreProcessLine{%
    \expandafter\ifstrempty\expandafter{\FV@Line}{}{%
      \FV@StepLineNo
      \FV@Gobble
      \expandafter\FV@ProcessLine\expandafter{\FV@Line}}}%
}
\makeatother


%%%%%%%%%%%%%%%%%%%%%%%%%%%%%%%%%%%%%%%%%%%%%%%%%%
% Reference (biber) settings
%%%%%%%%%%%%%%%%%%%%%%%%%%%%%%%%%%%%%%%%%%%%%%%%%%
\usepackage[style=verbose,backend=biber]{biblatex}
% \addbibresource{preamble/main.bib}
\addbibresource{\jobname.bib}
% \setbeamerfont{footnote}{size=\tiny}
\setbeamerfont{footnote}{size=\fontsize{4}{5}\selectfont}
\setbeamertemplate{bibliography item}{}% Remove reference icon.
\renewcommand*{\bibfont}{\footnotesize}


%%%%%%%%%%%%%%%%%%%%%%%%%%%%%%%%%%%%%%%%%%%%%%%%%%
% Colors
%%%%%%%%%%%%%%%%%%%%%%%%%%%%%%%%%%%%%%%%%%%%%%%%%%
\ifdefined\INVERTED
    \pagecolor{black} % Comment out to hide background
    \colorlet{primarycolor}{white}
    \colorlet{secondarycolor}{black}
    \colorlet{shadecolor}{black!85}
    \definecolor{cardinal}{HTML}{B83A4B}
    \definecolor{coolgrey}{HTML}{767674}
    \definecolor{captioncolor}{HTML}{59B3A9}
    \definecolor{captionsecondarycolor}{HTML}{2D716F}
    \colorlet{pagecolor}{lightgray}
    \colorlet{algbgcolor}{darkgray}
\else
    \colorlet{primarycolor}{black}
    \colorlet{secondarycolor}{white}
    \colorlet{shadecolor}{black!5}
    \definecolor{cardinal}{HTML}{8C1515}
    \definecolor{coolgrey}{HTML}{53565A}
    % \definecolor{captioncolor}{HTML}{3333B2} % Blue
    % \definecolor{captionsecondarycolor}{HTML}{7A7ACD} % Light blue
    \definecolor{captioncolor}{HTML}{59B3A9}
    \definecolor{captionsecondarycolor}{HTML}{2D716F}
    \colorlet{pagecolor}{darkgray}
    \colorlet{algbgcolor}{lightgray}
\fi

\definecolor{darkgreen}{RGB}{21,140,21}
\definecolor{darkblue}{RGB}{21,21,140}
\definecolor{sun}{RGB}{234,171,0}
\definecolor{repo}{HTML}{B6B1A9}
\definecolor{paloalto}{HTML}{2D716F}

\newcommand{\darkblue}[1]{{\color{darkblue} #1}}
\newcommand{\darkgreen}[1]{{\color{darkgreen} #1}}
\newcommand{\darkred}[1]{{\color{cardinal} #1}}

% Colorblind-Friendly Color Scheme
% A compromise between the original colorscheme and
% this palette: https://www.nature.com/articles/nmeth.1618
\definecolor{pastelMagenta}{HTML}{FF48CF} % magenta
\definecolor{pastelPurple}{HTML}{8770FE} % purple
\definecolor{pastelBlue}{RGB}{0,114,178} % blue 0072B2
\definecolor{pastelSkyBlue}{RGB}{86,180,233} % sky blue 56B4E9
\definecolor{pastelGreen}{RGB}{0,158,115} % green 009E73
\definecolor{pastelOrange}{RGB}{230,159,0} % orange E69F00
\definecolor{pastelRed}{HTML}{F5615C} % red
\definecolor{darkColor}{HTML}{300A24} % dark
\definecolor{pastelGreenBlue}{RGB}{43, 169, 174} % mix of pastelGreen and pastelSkyBlue

\newcommand{\colorPastelMagenta}{magenta}
\newcommand{\colorPastelPurple}{purple}
\newcommand{\colorPastelBlue}{blue}
\newcommand{\colorPastelSkyBlue}{sky blue}
\newcommand{\colorPastelGreen}{green}
\newcommand{\colorPastelOrange}{orange}
\newcommand{\colorPastelRed}{red}
\newcommand{\colorDark}{gray}


%%%%%%%%%%%%%%%%%%%%%%%%%%%%%%%%%%%%%%%%%%%%%%%%%%
% Beamer hide lines of code settings
%%%%%%%%%%%%%%%%%%%%%%%%%%%%%%%%%%%%%%%%%%%%%%%%%%
\tcbset{
  coloredbox/.style={
    colback=secondarycolor,
    colframe=secondarycolor,
    boxrule=0pt,
    width=\linewidth,
    sharp corners,
    height=0.83\baselineskip,
    % opacityback=0.5,
    before skip=0pt,
    after skip=0pt
  }
}

% Define a temporary length
\newlength{\testlen}
\newcommand{\lineblackout}[1]{%
    % Calculate the total shift: -\baselineskip - (N * 0.8)\baselineskip
    \pgfmathsetlength{\testlen}{-\baselineskip - #1 * 0.805\baselineskip}%
    \begin{textblock*}{\textwidth}(0cm, \testlen)
        \begin{tcolorbox}[standard jigsaw, coloredbox]\end{tcolorbox}
    \end{textblock*}
}


%%%%%%%%%%%%%%%%%%%%%%%%%%%%%%%%%%%%%%%%%%%%%%%%%%
% PGFPlot settings
%%%%%%%%%%%%%%%%%%%%%%%%%%%%%%%%%%%%%%%%%%%%%%%%%%
\usepackage{pgfplots}

\pgfplotsset{compat=newest}
\pgfplotsset{every axis legend/.append style={legend cell align=left, font=\footnotesize, draw=none, fill=none}}
\pgfplotsset{every axis/.append style={axis background/.style={fill=secondarycolor}}}
\pgfplotsset{every tick label/.append style={font=\footnotesize}}
\pgfplotsset{every axis label/.append style={font=\footnotesize}}

\fvset{baselinestretch=0.8}
\usepgfplotslibrary{fillbetween}
\usepgfplotslibrary{groupplots}
\usepgfplotslibrary{patchplots}
\usepgfplotslibrary{statistics}
\usepgfplotslibrary{ternary}


% NOTE: This won't work in Overleaf
\begin{jlcode}
  include("../../jl/support_code.jl")

  using Colors
  using ColorSchemes
  pasteljet = ColorMaps.RGBArrayMap(ColorSchemes.viridis, interpolation_levels=500, invert=true);
  pastelRedBlue = ColorMaps.RGBArrayMap([RGB(246/255, 21/255, 92/255),
                                         RGB(1.0,1.0,1.0),
                                         RGB( 27/255,161/255,234/255)], interpolation_levels=500);
\end{jlcode}


%%%%%%%%%%%%%%%%%%%%%%%%%%%%%%%%%%%%%%%%%%%%%%%%%%
% TikZ settings
%%%%%%%%%%%%%%%%%%%%%%%%%%%%%%%%%%%%%%%%%%%%%%%%%%
\usetikzlibrary{calc}
\usetikzlibrary{fit}
\usetikzlibrary{positioning}
\usetikzlibrary{arrows}
\usetikzlibrary{arrows.meta}
\usetikzlibrary{decorations.pathreplacing}
\usetikzlibrary{decorations.pathmorphing}
\usetikzlibrary{decorations.text}
\usetikzlibrary{patterns}
\usetikzlibrary{graphs}
\usetikzlibrary{graphdrawing}
\usetikzlibrary{shapes}
\usetikzlibrary{overlay-beamer-styles}

\pgfdeclarelayer{background}
\pgfdeclarelayer{main}
\pgfdeclarelayer{foreground}
\pgfdeclarelayer{top}
\pgfsetlayers{background, main, foreground, top}

\tikzset{
    func/.style = {rectangle, rounded corners=1, draw},
    partial/.style = {rectangle, darkgreen, font=\bfseries},
    input/.style = {rectangle},
    nnnode/.style = {circle, draw=primarycolor, fill=secondarycolor, minimum size=16pt,},
}

\tikzstyle{solid_line}=[solid, thick, mark=none]
\tikzset{every picture/.style={semithick, >=stealth'}}
\tikzset{myline/.style={line width = 0.05cm, rounded corners=5mm}}
\tikzset{myarrow/.style={line width = 0.05cm, ->, rounded corners=5mm}}
\tikzset{myaxis/.style={thick, ->, line cap=rect}}
\tikzset{roundednode/.style={rounded corners=4mm,draw=primarycolor,fill=secondarycolor,line width=0.05cm, minimum size=0.35in, align=center}}


%%%%%%%%%%%%%%%%%%%%%%%%%%%%%%%%%%%%%%%%%%%%%%%%%%
% Custom commands
%%%%%%%%%%%%%%%%%%%%%%%%%%%%%%%%%%%%%%%%%%%%%%%%%%
\newcommand{\smallcaps}[1]{\textsc{#1}}

\usepackage{mdframed}
\newenvironment{algorithmblock}[1][htbp]
{\begin{mdframed}[backgroundcolor=shadecolor,fontcolor=primarycolor,linecolor=primarycolor,rightline=false,leftline=false]}
{\end{mdframed}}

\newenvironment{definitionblock}[1]{%
    \begin{mdframed}[backgroundcolor=shadecolor,fontcolor=primarycolor,linecolor=primarycolor,rightline=false,leftline=false]
        \textbf{Definition: #1.}\;
}{%
    \end{mdframed}
}

\newenvironment{centerjuliaverbatim}{%
  \par
  \centering
  \varwidth{\linewidth}%
  \juliaverbatim
}{%
  \endjuliaverbatim
  \endvarwidth
  \par
}


%%%%%%%%%%%%%%%%%%%%%%%%%%%%%%%%%%%%%%%%%%%%%%%%%%
% Beamer settings
%%%%%%%%%%%%%%%%%%%%%%%%%%%%%%%%%%%%%%%%%%%%%%%%%%
\addtobeamertemplate{navigation symbols}{}{%
    % Page numbers
    \usebeamerfont{footline}%
    \usebeamercolor[fg]{footline}%
    \hspace{1em}%
    \insertframenumber/\inserttotalframenumber
}

% Fix right-hand line of non-black pixels in Preview
\setbeamertemplate{background}{%
    \begin{tikzpicture}[remember picture,overlay]
        \fill[secondarycolor] (current page.south west) rectangle (current page.north east);
    \end{tikzpicture}
}

\newcommand{\email}[1]{\def\@email{\texttt{\MakeLowercase{\textls[10]{#1}}}}}

\setbeamercolor{background canvas}{bg=secondarycolor} % Set background color
\setbeamercolor{normal text}{fg=primarycolor}       % Set foreground color

% \setbeamerfont{title}{series=\bfseries} % family=\sourcesanspro
\setbeamercolor{title}{fg=secondarycolor}
\setbeamerfont{subtitle}{series=\scshape} % family=\sourcesanspro
\setbeamercolor{subtitle}{fg=primarycolor}
\setbeamerfont{author}{series=\scshape}
\setbeamercolor{author}{fg=primarycolor}
\setbeamerfont{institute}{series=\scshape}
\setbeamercolor{institute}{fg=cardinal}
\setbeamerfont{email}{series=\mdseries,family=\sourcesanspro,size=\footnotesize}
\setbeamercolor{email}{fg=coolgrey}
\setbeamerfont{date}{family=\sourcesanspro,series=\scshape,size=\tiny}
\setbeamercolor{date}{fg=coolgrey}

\setbeamercolor{bibliography entry author}{fg=captioncolor} % Set author names
\setbeamercolor{bibliography entry note}{fg=captionsecondarycolor} % Set journal/publisher

\setbeamercolor{footline}{fg=pagecolor} % Page number color

\setbeamercolor{caption name}{fg=captioncolor} % "Figure" or "Table" label

\setbeamercolor{itemize item}{fg=primarycolor}
\setbeamercolor{itemize subitem}{fg=primarycolor}
\setbeamercolor{itemize subsubitem}{fg=primarycolor}
\setbeamercolor{section head}{fg=cardinal} % currently unused.

\setbeamertemplate{itemize item}[circle]
\setbeamertemplate{itemize subitem}{{\textendash}}
\setbeamertemplate{itemize subsubitem}[triangle]

\setbeamerfont{frametitle}{series=\scshape} % \itshape
\setbeamercolor{frametitle}{fg=primarycolor}

\setbeamertemplate{frametitle}
{
    \vspace*{0.7cm}
    \insertframetitle\par
      {\usebeamerfont{framesubtitle}\usebeamercolor[fg]{framesubtitle}\insertframesubtitle}

}

\AtBeginSection[]{
  \begin{frame}
  \vfill
  \centering
  \begin{beamercolorbox}[sep=8pt,center]{title}
    {\usebeamerfont{title}\usebeamercolor[fg]{title}{\textsc{\insertsectionhead}}}\par%
  \end{beamercolorbox}
  \vfill
  \end{frame}
}

%%%%%%%%%%%%%%%%%%%%%%%%%%%%%%%%%%%%%%%%%%%%%%%%%%
% Math definitions
%%%%%%%%%%%%%%%%%%%%%%%%%%%%%%%%%%%%%%%%%%%%%%%%%%
\newcommand{\dset}{\mathcal{D}}
\newcommand{\params}{\vect \theta}

\newcommand{\true}{\text{true}}
\newcommand{\false}{\text{false}}
\newcommand{\transpose}{\top}

\newcommand{\noisy}[1]{\tilde{#1}}

\newcommand{\mat}[1]{\vect{#1}}
\renewcommand{\vec}[1]{\vect{#1}}

\usepackage{mathtools}
\DeclarePairedDelimiter{\paren}{\lparen}{\rparen}
\DeclarePairedDelimiter{\brock}{\lbrack}{\rbrack}
\DeclarePairedDelimiter{\curly}{\{}{\}}
\DeclarePairedDelimiter{\norm}{\lVert}{\rVert}
\DeclarePairedDelimiter{\abs}{\lvert}{\rvert}
\DeclarePairedDelimiter{\anglebrackets}{\langle}{\rangle}
\DeclarePairedDelimiter{\ceil}{\lceil}{\rceil}
\DeclarePairedDelimiter{\floor}{\lfloor}{\rfloor}
\DeclarePairedDelimiter{\card}{|}{|}

\newcommand{\minimize}{\operatornamewithlimits{minimize}}
\newcommand{\maximize}{\operatornamewithlimits{maximize}}
\newcommand{\supremum}{\operatornamewithlimits{supremum}}
\newcommand{\argmin}{\operatornamewithlimits{arg\,min}}
\newcommand{\argmax}{\operatornamewithlimits{arg\,max}}
\newcommand{\subjectto}{\operatorname{subject~to}}
\newcommand{\for}{\text{for} \;}
\newcommand{\dimension}[1]{\text{dim}\paren*{#1}}
\newcommand{\gaussian}[2]{\mathcal{N}(#1, #2)}
\newcommand{\Gaussian}[2]{\mathcal{N}\paren*{#1, #2}}
\newcommand{\R}{\mathbb{R}}
\newcommand{\Z}{\mathbb{Z}}
\newcommand{\N}{\mathbb{N}}
\DeclareMathOperator{\sign}{sign}
\DeclareMathOperator{\Real}{\text{Re}}
\DeclareMathOperator{\Imag}{\text{Im}}
\DeclareMathOperator{\nil}{\textsc{nil}}
\DeclareMathOperator{\Expectation}{\mathbb{E}}
\DeclareMathOperator{\Variance}{\mathrm{Var}}
\DeclareMathOperator{\Normal}{\mathcal{N}}
\DeclareMathOperator{\Uniform}{\mathcal{U}}
\DeclareMathOperator{\Dirichlet}{Dir}
\DeclareMathOperator{\atantwo}{atan2}
\DeclareMathOperator{\modOne}{mod_1}
\DeclareMathOperator{\trace}{Tr}
\newcommand{\minprob}[3]{
\begin{aligned}
    \minimize_{#1} & & #2\\
    \subjectto & & #3 \\
\end{aligned}
}
\DeclareMathOperator{\Var}{Var}
\DeclareMathOperator{\SD}{SD}
\DeclareMathOperator{\Ber}{Ber}
\DeclareMathOperator{\Bin}{Bin}
\DeclareMathOperator{\Poi}{Poi}
\DeclareMathOperator{\Geo}{Geo}
\DeclareMathOperator{\NegBin}{NegBin}
\DeclareMathOperator{\Uni}{Uni}
\DeclareMathOperator{\Exp}{Exp}
\DeclareMathOperator{\Dir}{Dir}
\newcommand*\Eval[1]{\left.#1\right\rvert} % derivative/integration evaluation bar |
\DeclareMathOperator{\Cov}{Cov}
\DeclareMathOperator{\BetaDistribution}{Beta}
\DeclareMathOperator{\Beta}{Beta}
\DeclareMathOperator{\GammaDist}{Gamma}
\DeclareMathOperator{\Gumbel}{Gumbel}
\DeclareMathOperator{\Std}{Std}
\DeclareMathOperator{\Train}{\mathcal{D}_{\text{train}}}
\DeclareMathOperator{\Dtrain}{\mathcal{D}_{\text{train}}}
\DeclareMathOperator{\TrainLoss}{TrainLoss}
\DeclareMathOperator{\Loss}{Loss}
\DeclareMathOperator{\ZeroOneLoss}{Loss_{0\text{-}1}}
\DeclareMathOperator{\SquaredLoss}{Loss_{\text{squared}}}
\DeclareMathOperator{\AbsDevLoss}{Loss_{\text{absdev}}}
\DeclareMathOperator{\HingeLoss}{Loss_{\text{hinge}}}
\DeclareMathOperator{\LogisticLoss}{Loss_{\text{logistic}}}
\newcommand{\bfw}{\mathbf{w}}
\newcommand{\bbI}{\mathbb{I}}
\newcommand{\E}{\mathbb{E}}
\DeclareMathOperator{\Miss}{Miss}
\DeclareMathOperator{\sgn}{sgn}
\newcommand{\1}{\mathbb{1}}
\renewcommand{\v}{\mathbf{v}}
\newcommand{\V}{\mathbf{V}}
\newcommand{\w}{\mathbf{w}}
\newcommand{\h}{\mathbf{h}}
\newcommand{\opt}{*}
\DeclareMathOperator{\States}{States}
\DeclareMathOperator{\StartState}{s_{\text{state}}}
\DeclareMathOperator{\Actions}{Actions}
\DeclareMathOperator{\Reward}{Reward}
\DeclareMathOperator{\IsEnd}{IsEnd}
\DeclareMathOperator{\Cost}{Cost}
\DeclareMathOperator{\FutureCost}{FutureCost}
\DeclareMathOperator{\Succ}{Succ}
\DeclareMathOperator{\until}{\mathcal{U}}


%%%%%%%%%%%%%%%%%%%%%%%%%%%%%%%%%%%%%%%%%%%%%%%%%%
% Algorithm style.
%%%%%%%%%%%%%%%%%%%%%%%%%%%%%%%%%%%%%%%%%%%%%%%%%%
\renewcommand\algorithmicthen{} % Remove "then"
\renewcommand\algorithmicdo{} % Remove "do"


%%%%%%%%%%%%%%%%%%%%%%%%%%%%%%%%%%%%%%%%%%%%%%%%%%
% Auxiliary files
%%%%%%%%%%%%%%%%%%%%%%%%%%%%%%%%%%%%%%%%%%%%%%%%%%
\titlegraphic{\input{titleplot.tex}}

\defbeamertemplate*{title page}{customized}[1][]
{
    \ifdefined\CENTERTITLE\begin{center}\fi
    {\usebeamerfont{title}\textls[100]{\textbf{\inserttitle}}}\par
    {\usebeamerfont{subtitle}\usebeamercolor[fg]{subtitle}\textls[100]{\insertsubtitle}}\par\par
    \ifdefined\CENTERTITLE\end{center}\fi
    % \vfill
    \begin{center}
        {\usebeamercolor[fg]{titlegraphic}\inserttitlegraphic}
    \end{center}
    % \vfill
    \ifdefined\CENTERTITLE\begin{center}\fi
    {\usebeamerfont{author}\usebeamercolor[fg]{author}\textls[100]{\insertauthor}}\par
    {\usebeamerfont{institute}\usebeamercolor[fg]{institute}\textls[100]{\insertinstitute}}\par
    % \bigskip
    {\usebeamerfont{email}\usebeamercolor[fg]{email}\@email}\par
    \usebeamerfont{date}{\usebeamercolor[fg]{date}\textls[100]{\insertdate}}\par
    \ifdefined\CENTERTITLE\end{center}\fi
}

% Small overbrace
\makeatletter
\def\smalloverbrace#1{\mathop{\vbox{\m@th\ialign{##\crcr\noalign{\kern3\p@}%
    \tiny\downbracefill\crcr\noalign{\kern3\p@\nointerlineskip}%
    $\hfil\displaystyle{#1}\hfil$\crcr}}}\limits}
\makeatother

% Small underbrace
\makeatletter
\def\smallunderbrace#1{\mathop{\vtop{\m@th\ialign{##\crcr
    $\hfil\displaystyle{#1}\hfil$\crcr
    \noalign{\kern3\p@\nointerlineskip}%
    \tiny\upbracefill\crcr\noalign{\kern3\p@}}}}\limits}
\makeatother

\makeatletter
\newcommand{\smallgrayunderbrace}[2]{%
  \mathop{\vtop{\m@th\ialign{##\crcr
      $\hfil\displaystyle{#1}\hfil$\crcr
      \noalign{\kern3\p@\nointerlineskip}%
      \textcolor{gray}{\tiny\upbracefill}\crcr\noalign{\kern3\p@}}}}_{\color{gray}\substack{#2}}%
}
\makeatother

% Over and under arrows
\newcommand{\overarrow}[2]{\overset{\mathclap{\substack{#2 \\ \downarrow}}}{#1}}
\newcommand{\underarrow}[2]{\underset{\mathclap{\substack{\uparrow \\ #2}}}{#1}}
\newcommand{\undergrayarrow}[2]{\underset{\mathclap{\substack{\color{gray}\uparrow \\ {\color{gray}#2}}}}{#1}}


\title{\texttt{SignalTemporalLogic.jl}}
\subtitle{Introduction}
\author{Robert Moss}
\institute{Stanford AA228V/CS238V}
\email{mossr@cs.stanford.edu}
\date{}

\begin{document}

\begin{frame}
    \maketitle
\end{frame}

% [fragile] needed when the content contains juliaverbatim.
\begin{frame}[fragile]{Installation} \pause

You can install the \jlv{SignalTemporalLogic.jl} package via:

\begin{algorithmblock}
\begin{juliaverbatim}
using Pkg
Pkg.add("SignalTemporalLogic")
\end{juliaverbatim}
\end{algorithmblock}

\vfill \pause

Then you can run this to use the package:

\begin{algorithmblock}
\begin{juliaverbatim}
using SignalTemporalLogic
\end{juliaverbatim}
\end{algorithmblock}
    
\end{frame}
\include{slides/spec}
% Note, [fragile] is needed when using juliaconsole
\begin{frame}[fragile,t]{Specifications}

\phantom{}

Let's define the following \textit{specification} over a trajectory $\tau$:
\begin{equation*}
    \psi(\tau) = \lozenge\big( s_t > 0 \big) \tag*{\textit{``Eventually $(\lozenge)$, the state will be greater than zero.''}}
\end{equation*}

\begin{footnotesize}
\begin{juliaconsole}
using SignalTemporalLogic
τ = [-1.0, -3.2, 2.0, 1.5, 3.0, 0.5, -0.5, -2.0, -4.0, -1.5];
ψ = @formula ◊(sₜ -> sₜ > 0);
ψ(τ)
\end{juliaconsole}
\end{footnotesize}

\begin{tikzpicture}[remember picture, overlay]
    \begin{pgfonlayer}{top}
    \node[anchor=south east] at ($(current page.south east)+(0.5cm,0)$) {
        \begin{jlcode}
            p = let
                times = collect(1:10)
                τ = [-1.0, -3.2, 2.0, 1.5, 3.0, 0.5, -0.5, -2.0, -4.0, -1.5];
        
                xmin, xmax, ymin, ymax = 1, 10, -5, 4
                ax = Axis(xmin=xmin, xmax=xmax, ymin=ymin, ymax=ymax, width="9cm", height="4.5cm")
                ax.xlabel = "Time"
                ax.ylabel = L"s"
                # ax.title = raw"Signal $\tau$"
                # ax.style = raw"title style={font=\footnotesize}"
        
                # Used to add phantom right yaxis for better centering
                axr = deepcopy(ax)
                axr.style = raw", axis y line*=right, axis x line=none, yticklabel={\phantom{\pgfmathprintnumber{\tick}}}"
                axr.ylabel = "\\phantom{$(ax.ylabel)}"
                push!(axr, Plots.Command(""))
        
                push!(ax, Plots.Linear(times, τ, style="solid, mark=*, gray, line width=1pt, mark options={fill=gray!50, draw=gray}, mark size=2pt"))
                push!(ax, Plots.Linear([xmin, xmax], [0, 0], style="dotted, mark=none, gray, line width=1pt"))
                [axr, ax]
            end
            plot(p)
        \end{jlcode}
        \plot{fig/jl_spec2}
    };
    \end{pgfonlayer}
\end{tikzpicture}

{\footnotesize\jlv{τ # \tau<TAB>}}

{\footnotesize\jlv{ψ # \psi<TAB>}}

{\footnotesize\jlv{◊ # \lozenge<TAB>}}

\end{frame}
% Note, [fragile] is needed when using juliaconsole
\begin{frame}[fragile,t]{Robustness}

\phantom{}

You can also compute the \textit{robustness} of a trajectory $\tau$.

\phantom{}

\begin{footnotesize}
\begin{juliaconsole}
using SignalTemporalLogic
τ = [-1.0, -3.2, 2.0, 1.5, 3.0, 0.5, -0.5, -2.0, -4.0, -1.5];
ψ₁ = @formula ◊(sₜ -> sₜ > 0);
ρ₁ = ρ(τ, ψ₁)
ψ₂ = @formula □(sₜ -> sₜ > 0);
ρ₂ = ρ(τ, ψ₂)
\end{juliaconsole}
\onslide<1>{\lineblackout{7}\lineblackout{6}\lineblackout{5}}
\onslide<1-2>{\lineblackout{4}\lineblackout{3}}
\onslide<1-3>{\lineblackout{2}}
\onslide<1-4>{\lineblackout{1}\lineblackout{0}}
\onslide<5>{}
\end{footnotesize}

\phantom{}

\pause\pause
{\footnotesize\jlv{ρ # \rho<TAB>}}

\pause
{\footnotesize\jlv{□ # \square<TAB>}}

\onslide<6>{\begin{tikzpicture}[remember picture, overlay]
    \node[anchor=south east] at ($(current page.south east)+(0.5cm,0)$) {
        \input{fig/jl_robustness} % Use \input not \plot when reusing plots
    };
\end{tikzpicture}}

\end{frame}
% Note, [fragile] is needed when using juliaconsole
\begin{frame}[fragile,t]{Use in Projects}

Wrappers are provided in the textbook/projects:

\pause
\phantom{}

\textit{Linear temporal logic (LTL)}
\begin{footnotesize}
\begin{algorithmblock}
\begin{juliaverbatim}
struct LTLSpecification <: Specification
	formula # formula specified using SignalTemporalLogic.jl
end
evaluate(ψ::LTLSpecification, τ) = ψ.formula([step.s for step in τ])
\end{juliaverbatim}
\end{algorithmblock}
\end{footnotesize}

\pause
\phantom{}

\textit{Signal temporal logic (STL, includes time interval)}
\begin{footnotesize}
\begin{algorithmblock}
\begin{juliaverbatim}
struct STLSpecification <: Specification
    formula # formula specified using SignalTemporalLogic.jl
    I       # time interval (e.g. 3:10)
end
evaluate(ψ::STLSpecification, τ) = ψ.formula([step.s for step in τ[ψ.I]])
\end{juliaverbatim}
\end{algorithmblock}
\end{footnotesize}

\end{frame}
\def\gwone{\smallgrayunderbrace{\lozenge G}{\substack{\text{reaches}\\\text{goal}}}}
\def\gwtwo{\, \wedge \, \smallgrayunderbrace{\neg C \until G}{\substack{\text{reach}\\\text{checkpoint}\\\text{before goal}}}}
\def\gwthree{\, \wedge \, \smallgrayunderbrace{\square \neg F}{\substack{\text{always}\\\text{avoid}\\\text{obstacles}}}}
% Note, [fragile] is needed when using juliaconsole
\begin{frame}[fragile,t]{Grid World}

\begin{columns}
\begin{column}{0.6\textwidth}
    \begin{align*}
        \onslide<2->{F(s_t) &: \text{the state $s$ at time $t$ contains an obstacle} \\}
        \onslide<3->{G(s_t) &: \text{the state $s$ at time $t$ is the goal} \\}
        \onslide<4->{C(s_t) &: \text{the state $s$ at time $t$ is the checkpoint}}
    \end{align*}
    \[
      \only<1-4>{\phantom{\psi = \gwone\gwtwo\gwthree}}
      \only<5>{\psi = \gwone\phantom{\gwtwo\gwthree}}
      \only<6>{\psi = \gwone\gwtwo\phantom{\gwthree}}
      \only<7->{\psi = \gwone\gwtwo\gwthree}
    \]
\end{column}
\begin{column}{0.4\textwidth}
    \begin{jlcode}
        p = let
            env = GridWorld(tprob=0.80)
            τ_true = [(; s = [1, 1]), (; s = [2, 1]), (; s = [3, 1]), (; s = [4, 1]), (; s = [5, 1]), (; s = [6, 1]), (; s = [7, 1]), (; s = [8, 1]), (; s = [8, 2]), (; s = [8, 3]), (; s = [8, 4]), (; s = [8, 5]), (; s = [8, 6]), (; s = [8, 7]), (; s = [8, 8]), (; s = [7, 8])]
    
            ax = plot_grid_world(env; checkpoint_cells=[[8, 3]])
            plot_gw_trajectory!(ax, τ_true, color="pastelGreen!50!white", linewidth="1pt", include_marks=true, mark_color="pastelGreen!50!white")
            ax.width = "4.5cm"
            ax.height = "4.5cm"
            # ax.title = L"\psi = \texttt{true}"
            ax
        end
        plot(p)
        \end{jlcode}
        \begin{center}
            \plot{fig/ltl_checkpoint}
        \end{center}
    \end{column}
\end{columns}

\pause\pause\pause\pause\pause\pause\pause
\vspace{0.8\baselineskip}
\begin{small}
\begin{algorithmblock}
\begin{juliaverbatim}
F = @formula sₜ -> sₜ == [5, 5]
G = @formula sₜ -> sₜ == [7, 8]
C = @formula sₜ -> sₜ == [8, 3]
ψ = LTLSpecification(@formula ◊(G) ∧ 𝒰(¬G, C) ∧ □(¬F))
\end{juliaverbatim}
\end{algorithmblock}
\end{small}

\end{frame}
% Note, [fragile] is needed when using juliaconsole
\begin{frame}[fragile,t]{Continuum World}

{\small
\begin{table} % problem specs
\begin{adjustbox}{width=\columnwidth}
\begin{tabular}[t]{cll}
    \toprule
    System & Property & Implementation \\
    \midrule
    \begin{jlcode}
    p = let
        res = BSON.load("../data/cw_risk_neutral.bson")
        grid = res[:grid]
        Q = res[:Q]
        env = ContinuumWorld(Σ=0.2*I(2))
        sensor = IdealSensor()
        agent = InterpAgent(grid, Q)
        cw = System(agent, env, sensor)

        Random.seed!(0)
        τ = rollout(cw, d=20)

        ax = plot_cw(env)
        plot_cw_trajectory!(ax, τ, color="pastelGreen")
        ax.height = "4cm"
        ax.width = "4cm"
        ax.title = "Continuum World"
        TikzPicture(PGFPlots.tikzCode(ax), options="baseline=(current bounding box.center)")
    end
    plot(p)
    \end{jlcode}
    \plot{fig/cw_system} \vspace{0.3cm}
    & 
    \begin{tabular}{@{}l@{}} \textit{``Reach the goal without} \\ \textit{hitting the obstacle''} \\ $G(s_t)$: $s_t$ is in the goal region \\ $F(s_t)$: $s_t$ is in the obstacle region \\ $\psi = \lozenge G(s_t) \wedge \square \neg F(s_t)$\end{tabular}
    &
    \begin{tabular}{@{}l@{}}
    \small \jlv{G = @formula s->norm(s.-[6.5,7.5])≤0.5} \\
    \small \jlv{F = @formula s->norm(s.-[4.5,4.5])≤0.5} \\
    \small \jlv{ψ = @formula ◊(G) ∧ □(¬F)}
    \end{tabular} \\
    \bottomrule
\end{tabular}
\end{adjustbox}
\end{table}}

\end{frame}

% Note, [fragile] is needed when using juliaconsole
\begin{frame}[fragile,t]{Inverted Pendulum}

\begin{table} % problem specs
\begin{adjustbox}{width=\columnwidth}
\begin{tabular}[t]{cll}
    \toprule
    System & Property & Implementation \\
    \midrule
    \begin{tikzpicture}[baseline=(current bounding box.center)]
        \fill[pastelGreen!30] (0.707, 0.707) arc (45:135:1) -- cycle;
        \fill[pastelRed!30] (-0.707, 0.707) arc (135:405:1) -- cycle;
        \filldraw[fill=pastelGreen!30, draw=pastelGreen!30] (0, 0) -- (0.707, 0.707) -- (-0.707, 0.707) -- cycle;
        \draw[darkgray, dashed] (0, 0) -- (0, 1.1);
        \draw[darkgray, line width = 1.5pt] (0, 0) -- (0.42, 0.906);
        \draw[darkgray] (0, 0.5) arc (90:65:0.5);
        \draw[pastelRed, thick, dashed] (0, 0) -- (0.78, 0.78);
        \draw[pastelRed, thick, dashed] (0, 0) -- (-0.78, 0.78);
        \fill[darkgray] (0, 0) circle (0.1);
    
        \node[darkgray] at (0.15, 0.7) {\footnotesize $\theta$};
        \node[white] at (1.1, 1.0) {\footnotesize $\pi/4$};
        \node[white] at (-1.1, 1.0) {\footnotesize $-\pi/4$};
        \node at (0, 1.6) {Inverted Pendulum};
    \end{tikzpicture} \vspace{0.4cm} & \begin{tabular}{@{}l@{}} \textit{``Keep the pendulum balanced''} \\ $B(s_t)$: $|\theta_t| \le \pi/4$ \\ $\psi = \square B(s_t)$\end{tabular} & 
    \begin{tabular}{@{}l@{}}
    \small \jlv{B = @formula s->abs(s[1])≤π/4} \\
    \small \jlv{ψ = @formula □(B)}
    \end{tabular} \\
    \bottomrule
\end{tabular}
\end{adjustbox}
\end{table}

\end{frame}

\begin{jlcode}
    p = let
        res = BSON.load("../data/cas_policy.bson")
        grid = res[:grid]
        Q = res[:Q]
        agent = InterpAgent(grid, Q)
        env = CollisionAvoidance(Ds=Normal(0, 3))
        sensor = IdealSensor()
        cas = System(agent, env, sensor)

        Random.seed!(1)
        τ = rollout(cas, d=41)

        ax = cas_axis()
        plot_cas_traj!(ax, τ, color="pastelGreen")
        ax.xlabel = "Time (s)"
        ax.style *= ", xtick={0, 10, 20, 30, 40}, xticklabels={\\num{41}, \\num{31}, \\num{21}, \\num{11}, \\num{1}}, xtick pos=left, ytick pos=left"
        ax.height = "4cm"
        ax.width = "4cm"
        ax.title = "Aircraft Collision Avoidance"
        TikzPicture(PGFPlots.tikzCode(ax), options="baseline=(current bounding box.center)")
    end
    plot(p)
\end{jlcode}
\plotlater{fig/cas_system}

% Note, [fragile] is needed when using juliaconsole
\begin{frame}[fragile,t]{Aircraft Collision Avoidance}

{\small\begin{table} % problem specs
\begin{adjustbox}{width=\columnwidth}
\begin{tabular}[t]{cll}
    \toprule
    System & Property & Implementation \\
    \midrule
    \onslide<2->{\input{fig/cas_system}}
    &
    \onslide<3->{\begin{tabular}{@{}l@{}} \textit{``Ensure at least 50 meters relative} \\ \textit{altitude between 40 and 41 seconds''} \\ $S(s_t)$: $|h_t| \ge 50$ \\ $\psi = \square_{[40,41]} S$\end{tabular}}
    & 
    \onslide<4->{\begin{tabular}{@{}l@{}}
    \small \jlv{S = @formula s->abs(s[1])≥50} \\
    \small \jlv{ψ = @formula □(40:41, S)}
    \end{tabular}} \\
    \bottomrule
\end{tabular}
\end{adjustbox}
\end{table}}

\end{frame}

% [fragile] needed when the content contains juliaverbatim.
\begin{frame}[fragile]{Resources}

\begin{columns}
    \begin{column}{0.5\textwidth}
        \pause
        \begin{center}
            \includegraphics[width=4cm]{data/stl-qr.png}

            \phantom{}

            \small
            \textcolor{pastelBlue}{\texttt{github.com/sisl/SignalTemporalLogic.jl}}
        \end{center}
    \end{column}
    \begin{column}{0.5\textwidth}
        \pause
        \begin{center}
            \includegraphics[width=4cm]{data/stl-slides-qr.png}

            \phantom{}

            \small
            \textcolor{pastelGreen}{\texttt{github.com/mossr/STL-mini-lecture}}
        \end{center}
    \end{column}
\end{columns}
    
\end{frame}

\end{document}

% TODO: albatian
\usefonttheme{serif}
\usefonttheme{professionalfonts}
\usepackage{mathtools}
\usepackage[tracking=true]{microtype}

\usepackage{bold-extra}
\usepackage{realscripts}
\usepackage{amsmath,bm,amssymb}
\usepackage{amsthm}
\usepackage{bbm}
\usepackage{tikz}
\usepackage[group-digits=integer,group-minimum-digits=4,group-separator={,},detect-all]{siunitx}
\usepackage{algorithmicx}
\usepackage{algorithm}
\usepackage[noend]{algpseudocode}
\usepackage{xcolor}
\usepackage{multirow}
\usepackage{longtable,tabularx,booktabs}
\usepackage[flushleft]{threeparttable}
\usepackage{vector} % local
\usepackage{varwidth}
\usepackage{fancyvrb}
\usepackage{tcolorbox}
\usepackage{ifthen}
\usepackage{pagecolor}
\usepackage{textpos}
\usepackage{calc}
\usepackage{adjustbox}
\usepackage{svg}
\usepackage{caption}
\usepackage{verbatimbox}
\usepackage{multimedia}
\usepackage{xmpmulti}

%%%%%%%%%%%%%%%%%%%%%%%%%%%%%%%%%%%%%%%%%%%%%%%%%%
% Unicode settings
%%%%%%%%%%%%%%%%%%%%%%%%%%%%%%%%%%%%%%%%%%%%%%%%%%
% \setmonofont{DejaVu Sans Mono}[Scale=MatchLowercase]
% \usepackage{newunicodechar}
% \newfontface{\calligraphic}{Latin Modern Math}[Scale=0.85]

% https://juliamono.netlify.app/faq/#can_i_use_it_in_a_latex_document
\setmonofont{JuliaMono-Regular}[
	UprightFont = JuliaMono-Medium,
	BoldFont = JuliaMono-ExtraBold,
	ItalicFont = JuliaMono-Light,
    Contextuals = Alternate,
    Scale = MatchLowercase,
    Path = ./fonts/,
    Extension = .ttf
]

% Remove empty lines in juliaconsole
% https://github.com/gpoore/pythontex/issues/98#issuecomment-318078264
\makeatletter
\AtBeginEnvironment{juliaconsole}{%
  \def\FV@@PreProcessLine{%
    \expandafter\ifstrempty\expandafter{\FV@Line}{}{%
      \FV@StepLineNo
      \FV@Gobble
      \expandafter\FV@ProcessLine\expandafter{\FV@Line}}}%
}
\makeatother


%%%%%%%%%%%%%%%%%%%%%%%%%%%%%%%%%%%%%%%%%%%%%%%%%%
% Reference (biber) settings
%%%%%%%%%%%%%%%%%%%%%%%%%%%%%%%%%%%%%%%%%%%%%%%%%%
\usepackage[style=verbose,backend=biber]{biblatex}
% \addbibresource{preamble/main.bib}
\addbibresource{\jobname.bib}
% \setbeamerfont{footnote}{size=\tiny}
\setbeamerfont{footnote}{size=\fontsize{4}{5}\selectfont}
\setbeamertemplate{bibliography item}{}% Remove reference icon.
\renewcommand*{\bibfont}{\footnotesize}


%%%%%%%%%%%%%%%%%%%%%%%%%%%%%%%%%%%%%%%%%%%%%%%%%%
% Colors
%%%%%%%%%%%%%%%%%%%%%%%%%%%%%%%%%%%%%%%%%%%%%%%%%%
\ifdefined\INVERTED
    \pagecolor{black} % Comment out to hide background
    \colorlet{primarycolor}{white}
    \colorlet{secondarycolor}{black}
    \colorlet{shadecolor}{black!85}
    \definecolor{cardinal}{HTML}{B83A4B}
    \definecolor{coolgrey}{HTML}{767674}
    \definecolor{captioncolor}{HTML}{59B3A9}
    \definecolor{captionsecondarycolor}{HTML}{2D716F}
    \colorlet{pagecolor}{lightgray}
    \colorlet{algbgcolor}{darkgray}
\else
    \colorlet{primarycolor}{black}
    \colorlet{secondarycolor}{white}
    \colorlet{shadecolor}{black!5}
    \definecolor{cardinal}{HTML}{8C1515}
    \definecolor{coolgrey}{HTML}{53565A}
    % \definecolor{captioncolor}{HTML}{3333B2} % Blue
    % \definecolor{captionsecondarycolor}{HTML}{7A7ACD} % Light blue
    \definecolor{captioncolor}{HTML}{59B3A9}
    \definecolor{captionsecondarycolor}{HTML}{2D716F}
    \colorlet{pagecolor}{darkgray}
    \colorlet{algbgcolor}{lightgray}
\fi

\definecolor{darkgreen}{RGB}{21,140,21}
\definecolor{darkblue}{RGB}{21,21,140}
\definecolor{sun}{RGB}{234,171,0}
\definecolor{repo}{HTML}{B6B1A9}
\definecolor{paloalto}{HTML}{2D716F}

\newcommand{\darkblue}[1]{{\color{darkblue} #1}}
\newcommand{\darkgreen}[1]{{\color{darkgreen} #1}}
\newcommand{\darkred}[1]{{\color{cardinal} #1}}

% Colorblind-Friendly Color Scheme
% A compromise between the original colorscheme and
% this palette: https://www.nature.com/articles/nmeth.1618
\definecolor{pastelMagenta}{HTML}{FF48CF} % magenta
\definecolor{pastelPurple}{HTML}{8770FE} % purple
\definecolor{pastelBlue}{RGB}{0,114,178} % blue 0072B2
\definecolor{pastelSkyBlue}{RGB}{86,180,233} % sky blue 56B4E9
\definecolor{pastelGreen}{RGB}{0,158,115} % green 009E73
\definecolor{pastelOrange}{RGB}{230,159,0} % orange E69F00
\definecolor{pastelRed}{HTML}{F5615C} % red
\definecolor{darkColor}{HTML}{300A24} % dark
\definecolor{pastelGreenBlue}{RGB}{43, 169, 174} % mix of pastelGreen and pastelSkyBlue

\newcommand{\colorPastelMagenta}{magenta}
\newcommand{\colorPastelPurple}{purple}
\newcommand{\colorPastelBlue}{blue}
\newcommand{\colorPastelSkyBlue}{sky blue}
\newcommand{\colorPastelGreen}{green}
\newcommand{\colorPastelOrange}{orange}
\newcommand{\colorPastelRed}{red}
\newcommand{\colorDark}{gray}


%%%%%%%%%%%%%%%%%%%%%%%%%%%%%%%%%%%%%%%%%%%%%%%%%%
% Beamer hide lines of code settings
%%%%%%%%%%%%%%%%%%%%%%%%%%%%%%%%%%%%%%%%%%%%%%%%%%
\tcbset{
  coloredbox/.style={
    colback=secondarycolor,
    colframe=secondarycolor,
    boxrule=0pt,
    width=\linewidth,
    sharp corners,
    height=0.83\baselineskip,
    % opacityback=0.5,
    before skip=0pt,
    after skip=0pt
  }
}

% Define a temporary length
\newlength{\testlen}
\newcommand{\lineblackout}[1]{%
    % Calculate the total shift: -\baselineskip - (N * 0.8)\baselineskip
    \pgfmathsetlength{\testlen}{-\baselineskip - #1 * 0.805\baselineskip}%
    \begin{textblock*}{\textwidth}(0cm, \testlen)
        \begin{tcolorbox}[standard jigsaw, coloredbox]\end{tcolorbox}
    \end{textblock*}
}


%%%%%%%%%%%%%%%%%%%%%%%%%%%%%%%%%%%%%%%%%%%%%%%%%%
% PGFPlot settings
%%%%%%%%%%%%%%%%%%%%%%%%%%%%%%%%%%%%%%%%%%%%%%%%%%
\usepackage{pgfplots}

\pgfplotsset{compat=newest}
\pgfplotsset{every axis legend/.append style={legend cell align=left, font=\footnotesize, draw=none, fill=none}}
\pgfplotsset{every axis/.append style={axis background/.style={fill=secondarycolor}}}
\pgfplotsset{every tick label/.append style={font=\footnotesize}}
\pgfplotsset{every axis label/.append style={font=\footnotesize}}

\fvset{baselinestretch=0.8}
\usepgfplotslibrary{fillbetween}
\usepgfplotslibrary{groupplots}
\usepgfplotslibrary{patchplots}
\usepgfplotslibrary{statistics}
\usepgfplotslibrary{ternary}


% NOTE: This won't work in Overleaf
\begin{jlcode}
  include("../../jl/support_code.jl")

  using Colors
  using ColorSchemes
  pasteljet = ColorMaps.RGBArrayMap(ColorSchemes.viridis, interpolation_levels=500, invert=true);
  pastelRedBlue = ColorMaps.RGBArrayMap([RGB(246/255, 21/255, 92/255),
                                         RGB(1.0,1.0,1.0),
                                         RGB( 27/255,161/255,234/255)], interpolation_levels=500);
\end{jlcode}


%%%%%%%%%%%%%%%%%%%%%%%%%%%%%%%%%%%%%%%%%%%%%%%%%%
% TikZ settings
%%%%%%%%%%%%%%%%%%%%%%%%%%%%%%%%%%%%%%%%%%%%%%%%%%
\usetikzlibrary{calc}
\usetikzlibrary{fit}
\usetikzlibrary{positioning}
\usetikzlibrary{arrows}
\usetikzlibrary{arrows.meta}
\usetikzlibrary{decorations.pathreplacing}
\usetikzlibrary{decorations.pathmorphing}
\usetikzlibrary{decorations.text}
\usetikzlibrary{patterns}
\usetikzlibrary{graphs}
\usetikzlibrary{graphdrawing}
\usetikzlibrary{shapes}
\usetikzlibrary{overlay-beamer-styles}

\pgfdeclarelayer{background}
\pgfdeclarelayer{main}
\pgfdeclarelayer{foreground}
\pgfdeclarelayer{top}
\pgfsetlayers{background, main, foreground, top}

\tikzset{
    func/.style = {rectangle, rounded corners=1, draw},
    partial/.style = {rectangle, darkgreen, font=\bfseries},
    input/.style = {rectangle},
    nnnode/.style = {circle, draw=primarycolor, fill=secondarycolor, minimum size=16pt,},
}

\tikzstyle{solid_line}=[solid, thick, mark=none]
\tikzset{every picture/.style={semithick, >=stealth'}}
\tikzset{myline/.style={line width = 0.05cm, rounded corners=5mm}}
\tikzset{myarrow/.style={line width = 0.05cm, ->, rounded corners=5mm}}
\tikzset{myaxis/.style={thick, ->, line cap=rect}}
\tikzset{roundednode/.style={rounded corners=4mm,draw=primarycolor,fill=secondarycolor,line width=0.05cm, minimum size=0.35in, align=center}}


%%%%%%%%%%%%%%%%%%%%%%%%%%%%%%%%%%%%%%%%%%%%%%%%%%
% Custom commands
%%%%%%%%%%%%%%%%%%%%%%%%%%%%%%%%%%%%%%%%%%%%%%%%%%
\newcommand{\smallcaps}[1]{\textsc{#1}}

\usepackage{mdframed}
\newenvironment{algorithmblock}[1][htbp]
{\begin{mdframed}[backgroundcolor=shadecolor,fontcolor=primarycolor,linecolor=primarycolor,rightline=false,leftline=false]}
{\end{mdframed}}

\newenvironment{definitionblock}[1]{%
    \begin{mdframed}[backgroundcolor=shadecolor,fontcolor=primarycolor,linecolor=primarycolor,rightline=false,leftline=false]
        \textbf{Definition: #1.}\;
}{%
    \end{mdframed}
}

\newenvironment{centerjuliaverbatim}{%
  \par
  \centering
  \varwidth{\linewidth}%
  \juliaverbatim
}{%
  \endjuliaverbatim
  \endvarwidth
  \par
}


%%%%%%%%%%%%%%%%%%%%%%%%%%%%%%%%%%%%%%%%%%%%%%%%%%
% Beamer settings
%%%%%%%%%%%%%%%%%%%%%%%%%%%%%%%%%%%%%%%%%%%%%%%%%%
\addtobeamertemplate{navigation symbols}{}{%
    % Page numbers
    \usebeamerfont{footline}%
    \usebeamercolor[fg]{footline}%
    \hspace{1em}%
    \insertframenumber/\inserttotalframenumber
}

% Fix right-hand line of non-black pixels in Preview
\setbeamertemplate{background}{%
    \begin{tikzpicture}[remember picture,overlay]
        \fill[secondarycolor] (current page.south west) rectangle (current page.north east);
    \end{tikzpicture}
}

\newcommand{\email}[1]{\def\@email{\texttt{\MakeLowercase{\textls[10]{#1}}}}}

\setbeamercolor{background canvas}{bg=secondarycolor} % Set background color
\setbeamercolor{normal text}{fg=primarycolor}       % Set foreground color

% \setbeamerfont{title}{series=\bfseries} % family=\sourcesanspro
\setbeamercolor{title}{fg=secondarycolor}
\setbeamerfont{subtitle}{series=\scshape} % family=\sourcesanspro
\setbeamercolor{subtitle}{fg=primarycolor}
\setbeamerfont{author}{series=\scshape}
\setbeamercolor{author}{fg=primarycolor}
\setbeamerfont{institute}{series=\scshape}
\setbeamercolor{institute}{fg=cardinal}
\setbeamerfont{email}{series=\mdseries,family=\sourcesanspro,size=\footnotesize}
\setbeamercolor{email}{fg=coolgrey}
\setbeamerfont{date}{family=\sourcesanspro,series=\scshape,size=\tiny}
\setbeamercolor{date}{fg=coolgrey}

\setbeamercolor{bibliography entry author}{fg=captioncolor} % Set author names
\setbeamercolor{bibliography entry note}{fg=captionsecondarycolor} % Set journal/publisher

\setbeamercolor{footline}{fg=pagecolor} % Page number color

\setbeamercolor{caption name}{fg=captioncolor} % "Figure" or "Table" label

\setbeamercolor{itemize item}{fg=primarycolor}
\setbeamercolor{itemize subitem}{fg=primarycolor}
\setbeamercolor{itemize subsubitem}{fg=primarycolor}
\setbeamercolor{section head}{fg=cardinal} % currently unused.

\setbeamertemplate{itemize item}[circle]
\setbeamertemplate{itemize subitem}{{\textendash}}
\setbeamertemplate{itemize subsubitem}[triangle]

\setbeamerfont{frametitle}{series=\scshape} % \itshape
\setbeamercolor{frametitle}{fg=primarycolor}

\setbeamertemplate{frametitle}
{
    \vspace*{0.7cm}
    \insertframetitle\par
      {\usebeamerfont{framesubtitle}\usebeamercolor[fg]{framesubtitle}\insertframesubtitle}

}

\AtBeginSection[]{
  \begin{frame}
  \vfill
  \centering
  \begin{beamercolorbox}[sep=8pt,center]{title}
    {\usebeamerfont{title}\usebeamercolor[fg]{title}{\textsc{\insertsectionhead}}}\par%
  \end{beamercolorbox}
  \vfill
  \end{frame}
}

%%%%%%%%%%%%%%%%%%%%%%%%%%%%%%%%%%%%%%%%%%%%%%%%%%
% Math definitions
%%%%%%%%%%%%%%%%%%%%%%%%%%%%%%%%%%%%%%%%%%%%%%%%%%
\newcommand{\dset}{\mathcal{D}}
\newcommand{\params}{\vect \theta}

\newcommand{\true}{\text{true}}
\newcommand{\false}{\text{false}}
\newcommand{\transpose}{\top}

\newcommand{\noisy}[1]{\tilde{#1}}

\newcommand{\mat}[1]{\vect{#1}}
\renewcommand{\vec}[1]{\vect{#1}}

\usepackage{mathtools}
\DeclarePairedDelimiter{\paren}{\lparen}{\rparen}
\DeclarePairedDelimiter{\brock}{\lbrack}{\rbrack}
\DeclarePairedDelimiter{\curly}{\{}{\}}
\DeclarePairedDelimiter{\norm}{\lVert}{\rVert}
\DeclarePairedDelimiter{\abs}{\lvert}{\rvert}
\DeclarePairedDelimiter{\anglebrackets}{\langle}{\rangle}
\DeclarePairedDelimiter{\ceil}{\lceil}{\rceil}
\DeclarePairedDelimiter{\floor}{\lfloor}{\rfloor}
\DeclarePairedDelimiter{\card}{|}{|}

\newcommand{\minimize}{\operatornamewithlimits{minimize}}
\newcommand{\maximize}{\operatornamewithlimits{maximize}}
\newcommand{\supremum}{\operatornamewithlimits{supremum}}
\newcommand{\argmin}{\operatornamewithlimits{arg\,min}}
\newcommand{\argmax}{\operatornamewithlimits{arg\,max}}
\newcommand{\subjectto}{\operatorname{subject~to}}
\newcommand{\for}{\text{for} \;}
\newcommand{\dimension}[1]{\text{dim}\paren*{#1}}
\newcommand{\gaussian}[2]{\mathcal{N}(#1, #2)}
\newcommand{\Gaussian}[2]{\mathcal{N}\paren*{#1, #2}}
\newcommand{\R}{\mathbb{R}}
\newcommand{\Z}{\mathbb{Z}}
\newcommand{\N}{\mathbb{N}}
\DeclareMathOperator{\sign}{sign}
\DeclareMathOperator{\Real}{\text{Re}}
\DeclareMathOperator{\Imag}{\text{Im}}
\DeclareMathOperator{\nil}{\textsc{nil}}
\DeclareMathOperator{\Expectation}{\mathbb{E}}
\DeclareMathOperator{\Variance}{\mathrm{Var}}
\DeclareMathOperator{\Normal}{\mathcal{N}}
\DeclareMathOperator{\Uniform}{\mathcal{U}}
\DeclareMathOperator{\Dirichlet}{Dir}
\DeclareMathOperator{\atantwo}{atan2}
\DeclareMathOperator{\modOne}{mod_1}
\DeclareMathOperator{\trace}{Tr}
\newcommand{\minprob}[3]{
\begin{aligned}
    \minimize_{#1} & & #2\\
    \subjectto & & #3 \\
\end{aligned}
}
\DeclareMathOperator{\Var}{Var}
\DeclareMathOperator{\SD}{SD}
\DeclareMathOperator{\Ber}{Ber}
\DeclareMathOperator{\Bin}{Bin}
\DeclareMathOperator{\Poi}{Poi}
\DeclareMathOperator{\Geo}{Geo}
\DeclareMathOperator{\NegBin}{NegBin}
\DeclareMathOperator{\Uni}{Uni}
\DeclareMathOperator{\Exp}{Exp}
\DeclareMathOperator{\Dir}{Dir}
\newcommand*\Eval[1]{\left.#1\right\rvert} % derivative/integration evaluation bar |
\DeclareMathOperator{\Cov}{Cov}
\DeclareMathOperator{\BetaDistribution}{Beta}
\DeclareMathOperator{\Beta}{Beta}
\DeclareMathOperator{\GammaDist}{Gamma}
\DeclareMathOperator{\Gumbel}{Gumbel}
\DeclareMathOperator{\Std}{Std}
\DeclareMathOperator{\Train}{\mathcal{D}_{\text{train}}}
\DeclareMathOperator{\Dtrain}{\mathcal{D}_{\text{train}}}
\DeclareMathOperator{\TrainLoss}{TrainLoss}
\DeclareMathOperator{\Loss}{Loss}
\DeclareMathOperator{\ZeroOneLoss}{Loss_{0\text{-}1}}
\DeclareMathOperator{\SquaredLoss}{Loss_{\text{squared}}}
\DeclareMathOperator{\AbsDevLoss}{Loss_{\text{absdev}}}
\DeclareMathOperator{\HingeLoss}{Loss_{\text{hinge}}}
\DeclareMathOperator{\LogisticLoss}{Loss_{\text{logistic}}}
\newcommand{\bfw}{\mathbf{w}}
\newcommand{\bbI}{\mathbb{I}}
\newcommand{\E}{\mathbb{E}}
\DeclareMathOperator{\Miss}{Miss}
\DeclareMathOperator{\sgn}{sgn}
\newcommand{\1}{\mathbb{1}}
\renewcommand{\v}{\mathbf{v}}
\newcommand{\V}{\mathbf{V}}
\newcommand{\w}{\mathbf{w}}
\newcommand{\h}{\mathbf{h}}
\newcommand{\opt}{*}
\DeclareMathOperator{\States}{States}
\DeclareMathOperator{\StartState}{s_{\text{state}}}
\DeclareMathOperator{\Actions}{Actions}
\DeclareMathOperator{\Reward}{Reward}
\DeclareMathOperator{\IsEnd}{IsEnd}
\DeclareMathOperator{\Cost}{Cost}
\DeclareMathOperator{\FutureCost}{FutureCost}
\DeclareMathOperator{\Succ}{Succ}
\DeclareMathOperator{\until}{\mathcal{U}}


%%%%%%%%%%%%%%%%%%%%%%%%%%%%%%%%%%%%%%%%%%%%%%%%%%
% Algorithm style.
%%%%%%%%%%%%%%%%%%%%%%%%%%%%%%%%%%%%%%%%%%%%%%%%%%
\renewcommand\algorithmicthen{} % Remove "then"
\renewcommand\algorithmicdo{} % Remove "do"


%%%%%%%%%%%%%%%%%%%%%%%%%%%%%%%%%%%%%%%%%%%%%%%%%%
% Auxiliary files
%%%%%%%%%%%%%%%%%%%%%%%%%%%%%%%%%%%%%%%%%%%%%%%%%%
\titlegraphic{\begin{figure}
    \begin{jlcode}
    p = let
        times = collect(1:10)
        τ = [-1.0, -3.2, 2.0, 1.5, 3.0, 0.5, -0.5, -2.0, -4.0, -1.5];

        xmin, xmax, ymin, ymax = 1, 10, -5, 4
        ax = Axis(xmin=xmin, xmax=xmax, ymin=ymin, ymax=ymax, width="9cm", height="4.5cm")
        ax.xlabel = "Time"
        ax.ylabel = L"s"
        ax.title = raw"${\color{pastelSkyBlue}\psi_1 = \lozenge\big( s_t > 0 \big)} \qquad {\color{pastelGreen}\psi_2 = \square\big( s_t > 0 \big)}$"
        ax.style = raw"title style={font=\footnotesize}"

        # Used to add phantom right yaxis for better centering
        axr = deepcopy(ax)
        axr.style *= raw", axis y line*=right, axis x line=none, yticklabel={\phantom{\pgfmathprintnumber{\tick}}}"
        axr.ylabel = "\\phantom{$(ax.ylabel)}"
        push!(axr, Plots.Command(""))

        push!(ax, Plots.Linear(times, τ, style="solid, mark=*, gray, line width=1pt, mark options={fill=gray!50, draw=gray}, mark size=2pt"))
        push!(ax, Plots.Linear([xmin, xmax], [0, 0], style="dotted, mark=none, gray, line width=1pt"))
        push!(ax, Plots.Scatter([5], [3], style="only marks, mark=*, mark size=2pt, mark options={fill=pastelSkyBlue!50, draw=pastelSkyBlue}"))
        push!(ax, Plots.Linear([xmin, xmax], [3, 3], style="dashed, mark=none, pastelSkyBlue, line width=1pt"))
        push!(ax, Plots.Command("\\node[anchor=center] at (axis cs: 8, 2.3) {\\scriptsize \\textcolor{pastelSkyBlue}{\$\\rho_1\$}};"))
        push!(ax, Plots.Scatter([9], [-4], style="only marks, mark=*, mark size=2pt, mark options={fill=pastelGreen!50, draw=pastelGreen}"))
        push!(ax, Plots.Linear([xmin, xmax], [-4, -4], style="dashed, mark=none, pastelGreen, line width=1pt"))
        push!(ax, Plots.Command("\\node[anchor=center] at (axis cs: 5, -3.3) {\\scriptsize \\textcolor{pastelGreen}{\$\\rho_2\$}};"))
        [axr, ax]
    end
    plot(p)
    \end{jlcode}
    \plot{fig/jl_robustness}
\end{figure}}

\defbeamertemplate*{title page}{customized}[1][]
{
    \ifdefined\CENTERTITLE\begin{center}\fi
    {\usebeamerfont{title}\textls[100]{\textbf{\inserttitle}}}\par
    {\usebeamerfont{subtitle}\usebeamercolor[fg]{subtitle}\textls[100]{\insertsubtitle}}\par\par
    \ifdefined\CENTERTITLE\end{center}\fi
    % \vfill
    \begin{center}
        {\usebeamercolor[fg]{titlegraphic}\inserttitlegraphic}
    \end{center}
    % \vfill
    \ifdefined\CENTERTITLE\begin{center}\fi
    {\usebeamerfont{author}\usebeamercolor[fg]{author}\textls[100]{\insertauthor}}\par
    {\usebeamerfont{institute}\usebeamercolor[fg]{institute}\textls[100]{\insertinstitute}}\par
    % \bigskip
    {\usebeamerfont{email}\usebeamercolor[fg]{email}\@email}\par
    \usebeamerfont{date}{\usebeamercolor[fg]{date}\textls[100]{\insertdate}}\par
    \ifdefined\CENTERTITLE\end{center}\fi
}

% Small overbrace
\makeatletter
\def\smalloverbrace#1{\mathop{\vbox{\m@th\ialign{##\crcr\noalign{\kern3\p@}%
    \tiny\downbracefill\crcr\noalign{\kern3\p@\nointerlineskip}%
    $\hfil\displaystyle{#1}\hfil$\crcr}}}\limits}
\makeatother

% Small underbrace
\makeatletter
\def\smallunderbrace#1{\mathop{\vtop{\m@th\ialign{##\crcr
    $\hfil\displaystyle{#1}\hfil$\crcr
    \noalign{\kern3\p@\nointerlineskip}%
    \tiny\upbracefill\crcr\noalign{\kern3\p@}}}}\limits}
\makeatother

\makeatletter
\newcommand{\smallgrayunderbrace}[2]{%
  \mathop{\vtop{\m@th\ialign{##\crcr
      $\hfil\displaystyle{#1}\hfil$\crcr
      \noalign{\kern3\p@\nointerlineskip}%
      \textcolor{gray}{\tiny\upbracefill}\crcr\noalign{\kern3\p@}}}}_{\color{gray}\substack{#2}}%
}
\makeatother

% Over and under arrows
\newcommand{\overarrow}[2]{\overset{\mathclap{\substack{#2 \\ \downarrow}}}{#1}}
\newcommand{\underarrow}[2]{\underset{\mathclap{\substack{\uparrow \\ #2}}}{#1}}
\newcommand{\undergrayarrow}[2]{\underset{\mathclap{\substack{\color{gray}\uparrow \\ {\color{gray}#2}}}}{#1}}
